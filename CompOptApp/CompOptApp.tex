%%%%%%%%%%%%%%%%%%%%%%% file template.tex %%%%%%%%%%%%%%%%%%%%%%%%%
%
% This is a general template file for the LaTeX package SVJour3
% for Springer journals.          Springer Heidelberg 2010/09/16
%
% Copy it to a new file with a new name and use it as the basis
% for your article. Delete % signs as needed.
%
% This template includes a few options for different layouts and
% content for various journals. Please consult a previous issue of
% your journal as needed.
%
%%%%%%%%%%%%%%%%%%%%%%%%%%%%%%%%%%%%%%%%%%%%%%%%%%%%%%%%%%%%%%%%%%%
%
% First comes an example EPS file -- just ignore it and
% proceed on the \documentclass line
% your LaTeX will extract the file if required
\begin{filecontents*}{example.eps}
%!PS-Adobe-3.0 EPSF-3.0
%%BoundingBox: 19 19 221 221
%%CreationDate: Mon Sep 29 1997
%%Creator: programmed by hand (JK)
%%EndComments
gsave
newpath
  20 20 moveto
  20 220 lineto
  220 220 lineto
  220 20 lineto
closepath
2 setlinewidth
gsave
  .4 setgray fill
grestore
stroke
grestore
\end{filecontents*}
%
\RequirePackage{fix-cm}
%
%\documentclass{svjour3}                     % onecolumn (standard format)
%\documentclass[smallcondensed]{svjour3}     % onecolumn (ditto)
\documentclass[smallextended]{svjour3}       % onecolumn (second format)
%\documentclass[twocolumn]{svjour3}          % twocolumn
%
\smartqed  % flush right qed marks, e.g. at end of proof
%
\usepackage{graphicx}
%
%
%
%\documentclass[envcountsame,envcountchap]{svmono}
%
\usepackage[latin1]{inputenc}
\usepackage[T1]{fontenc}
%\usepackage{graphicx}
\usepackage[bottom]{footmisc}
% places footnotes at page bottom
\usepackage{algorithmic}
\usepackage{algorithm}



\newcommand{\icm}{\emph{ICM}}
\newcommand{\ibm}{\emph{IBM}}
\newcommand{\prog}{\emph{3DHit}}



\begin{document}


\title{3D-Hit: fast structural comparison of proteins on multicore architectures}
\label{DPlewczynski} % Always give a unique label
%\thanks{Grants or other notes
%about the article that should go on the front page should be
%placed here. General acknowledgments should be placed at the end of the article.}
%}
%\subtitle{Do you have a subtitle?\\ If so, write it here}

%\titlerunning{Short form of title}        % if too long for running head

\author{\L{}. Bieniasz-Krzywiec, \and 
M. Cytowski, \and 
L. Rychlewski, \and
D. Plewczynski       
%        Second Author %etc.
}

%\authorrunning{Short form of author list} % if too long for running head


%$^a$ Interdisciplinary Centre for Mathematical and Computational Modelling, \\
%University of Warsaw (ICM), 
%Pawi\'nskiego 5a, 02-106 Warsaw, Poland \\
%$^b$ Faculty of Mathematics, Informatics and Mechanics, University of Warsaw,\\ Banacha 2, 02-097 Warsaw, Poland, \\
%$^c$ BioInfoBank Institute, Limanowskiego 24A16, 60-744 Pozna\'n, Poland \\


\institute{D. Plewczynski, \L{}. Bieniasz-Krzywiec, M. Cytowski \at
Interdisciplinary Centre for Mathematical and Computational Modelling, \\
University of Warsaw (ICM),  \\
Pawi\'nskiego 5a, 02-106 Warsaw, Poland \\
              \email{darman@icm.edu.pl}           %  \\
%             \emph{Present address:} of F. Author  %  if needed
           \and
           L. Rychlewski \at
BioInfoBank Institute, \\
Limanowskiego 24A16, \\ 
60-744 Pozna\'n, Poland \\
}

\date{Received: date / Accepted: date}
% The correct dates will be entered by the editor


\maketitle

%
%
%
%
%
%\chapter{3D-Hit: fast structural comparison of proteins on multicore architectures}
%\label{DPlewczynski} % Always give a unique label
%
%
%\begin{flushright}
%-- Arxiv paper -- \\
%\end{flushright}
%
%\hspace{20 mm}
%
%\begin{center}
%\L{}. Bieniasz-Krzywiec$^a$$^,$$^b$, M. Cytowski$^a$, L. Rychlewski$^c$, D. Plewczynski$^a$ \\
%\end{center}
%\hspace{20 mm}
%
%\begin{flushleft} 
%$^a$ Interdisciplinary Centre for Mathematical and Computational Modelling, \\
%University of Warsaw (ICM), 
%Pawi\'nskiego 5a, 02-106 Warsaw, Poland \\
%$^b$ Faculty of Mathematics, Informatics and Mechanics, University of Warsaw,\\ Banacha 2, 02-097 Warsaw, Poland, \\
%$^c$ BioInfoBank Institute, Limanowskiego 24A16, 60-744 Pozna\'n, Poland \\
%\end{flushleft}

% \title{3D-Hit: fast structural comparison of proteins on multicore architectures}

%\address[ICM]{Interdisciplinary Centre for Mathematical and Computational Modelling, University of Warsaw (ICM),\\ Pawi\'nskiego 5a, 02-106 Warsaw, Poland}
%\address[MIMUW]{Faculty of Mathematics, Informatics and Mechanics, University of Warsaw,\\ Banacha 2, 02-097 Warsaw, Poland},
%\address[BioInfo]{BioInfoBank Institute,\\ Limanowskiego 24A16, 60-744 Pozna\'n, Poland},

%\author{\L{}. Bieniasz-Krzywiec {^{\it a,b}}, \\
%M. Cytowski {^{\it a}}, \\
%L. Rychlewski {^{\it c}},   \\     
%D. Plewczynski {^{\it a}}
%}


%\author{\L{}. Bieniasz-Krzywiec$^a$$^,$$^b$, M. Cytowski$^a$, L. Rychlewski$^c$, D. Plewczynski$^a$}
%\title{3D-Hit: fast structural comparison of proteins on multicore architectures\\
%{\small $^a$ Interdisciplinary Centre for Mathematical and Computational Modelling, \\
%University of Warsaw (ICM), 
%Pawi\'nskiego 5a, 02-106 Warsaw, Poland \\
%$^b$ Faculty of Mathematics, Informatics and Mechanics, University of Warsaw,\\ Banacha 2, 02-097 Warsaw, Poland, \\
%$^c$ BioInfoBank Institute,\\ Limanowskiego 24A16, 60-744 Pozna\'n, Poland }}
%
%
%\subtitle{-- PPAM 2011 conference paper --}
%\maketitle
%
%
% \maketitle

%%%%%%%%%%%%%%%%%%%%%%%%%%%%%%%%%%%%%%%%%%%%%%%%%%%%%%%%%%%%%%%%%%%%%%%%%%%%%%%%%%%%%%%%%%%%%%%%%%%%%%%%%%%%
\begin{abstract}
%%%%%%%%%%%%%%%%%%%%%%%%%%%%%%%%%%%%%%%%%%%%%%%%%%%%%%%%%%%%%%%%%%%%%%%%%%%%%%%%%%%%%%%%%%%%%%%%%%%%%%%%%%%%

3D-Hit is well established method for rapid detection of structural similarities between proteins, which 
is widely used in various bioinformatics web servers
(MetaServer, GRDB, 3D-Fun, Rosetta, etc.). 
The algorithm decomposes proteins into set of overlaping segments of 9-13 residues, then tries to
match them using root mean square distance metric. The best aligned pairs of segments are selected as seeds for futher analysis. 
Those initial hits are expanded by iterative process in order to construct the global structural alignment by 
concatenating pairs of matching segments.
The method has the same accuracy as the other state-of-the-art structural comparison algorithms (LGscore2, DALI), yet it provides
much faster procesing times, and can be used in the high-throughput setup as the structural module in bioinformatics pipelines. 
The method is optimized in terms of speed and accuracy to work on  novel computer architectures, such as PowerXCell8i and Sun Constellation System. Here, we provide the source code of the 3D-Hit program, describe selected architectures on which the software was ported, 
present programing models, point out significant porting steps
and sumarize performance comparisons.

\keywords{proteins \and IBM Cell \and structure comparison \and bioinformatics \and optimization}
% \PACS{PACS code1 \and PACS code2 \and more}
% \subclass{MSC code1 \and MSC code2 \and more}


\end{abstract}

%%%%%%%%%%%%%%%%%%%%%%%%%%%%%%%%%%%%%%%%%%%%%%%%%%%%%%%%%%%%%%%%%%%%%%%%%%%%%%%%%%%%%%%%%%%%%%%%%%%%%%%%%%%%
\section{Introduction}
%%%%%%%%%%%%%%%%%%%%%%%%%%%%%%%%%%%%%%%%%%%%%%%%%%%%%%%%%%%%%%%%%%%%%%%%%%%%%%%%%%%%%%%%%%%%%%%%%%%%%%%%%%%%
The PowerXCell8i is a pioneering microprocessor architecture
is supposed to bridge the gap between conventional desktop computers and
specialized high-performance machines.
It has been designed to support very wide range of applications.
The Cell processor consists of nine processor elements operating on a
shared and coherent memory.
Functions of those processors are specialized into two types.
Cell has got one \emph{PowerPC Processing Element} (PPE), which usually acts
as a controller and is designed to run operating system.
It also includes eight \emph{Synergistic Processing Elements} (SPEs), which
handle computational workload and are optimized for SIMD code execution.
PowerXCell8i processors are usually available in a dual processor nodes within
the IBM QS22 blades.
This gives a shared memory programming environment with 16
computational cores.
Moreover due to its specific architecture, Cell processor achieves very good
performance results with relatively low power consumption (in terms of
performance per Watt).
As a result, \emph{Nautilus} \cite{nau} --- cluster of IBM QS22 blades based on
Cell processors installed at \emph{Interdisciplinary Centre for Mathematical and
Computational Modelling (ICM)} has been ranked as No. 1 in two editions of
The Green500 List \cite{g500} of the world most energy-efficient supercomputers.\\
The Sun Constellation System, another high-performance computer available at ICM,
is a cluster based on a x86 architecture.
This powerful machine consists of number of blades packed into special-purpose
racks, which are tied together with highly-efficient InfiniBand connections.
Each blade is equipped with four AMD Quad-Core Opteron 835X processors (which
gives 16 computational cores per blade) and up to 32 Gb of memory.\\
In this paper we describe two fast implementations of an efficient scanning
method for detecting structural similarities between proteins.
The first of them is application designed for the PowerXCell8i processors.
The second takes advantage of OpenMP and therefore can be executed on
virtually all architectures providing shared memory access, including the Sun
Constellation System.
As a result we are able to compare two shared memory systems with different
architectures but the same number of computational cores.
Algorithm used in our programs  was originally created by Dariusz Plewczy\'nski
et al. \cite{3dhit1,3dhit2}\\
The original code, destined to execute on x86 architecture, was ported using two
different frameworks: Cell SDK with its SPE library (for PowerXCell8i
architecture) and OpenMP (for parallel computers with shared memory).
The Cell-accelerated application achieved an overall speedup of 12 over
single threaded version executing on 1 SPE core.
This level of performance was obtained with the use of all 16 SPU cores
available within IBM QS22 blade.
Program parallelized with OpenMP library performed even better in terms of the
final walltime achieved during benchmarking tests.
In the course of our work we encountered very interesting aspects of parallel
programming and learned how to identity parts of code whose performance
could benefit from novel high-performance architectures.

%%%%%%%%%%%%%%%%%%%%%%%%%%%%%%%%%%%%%%%%%%%%%%%%%%%%%%%%%%%%%%%%%%%%%%%%%%%%%%%%%%%%%%%%%%%%%%%%%%%%%%%%%%%%
\section{Structural alignment}
%%%%%%%%%%%%%%%%%%%%%%%%%%%%%%%%%%%%%%%%%%%%%%%%%%%%%%%%%%%%%%%%%%%%%%%%%%%%%%%%%%%%%%%%%%%%%%%%%%%%%%%%%%%%
It has been discovered that three dimensional structure of a protein is more
conserved during the process of evolution than its sequence alone \cite{chle}.
Therefore, the comparison of 3D structures of two proteins makes it possible to
establish distant evolutionary relationships, even between very diverged proteins.
As a result, the structural alignment of proteins increases our understanding of
more distant evolutionary relationships \cite{buj2000,jms}.
The correspondence between structural and functional classification enables
scientists to determine functions of various newly discovered folds and whole
protein families. Structural similarity can suggest evolutionary links between protein
families, which can result in more detailed functional annotation
of given protein on molecular level.
Moreover, the structural comparison can guide the experimental structure
determination process, by tracing shifts in low resolution
models.
The aforementioned reasons make the structural alignment the 
very important part of bioinformatics every-day work.\\
On the other hand, the size of Protein Data Bank \cite{pdb} 
is growing rapidly doubling every 18 months.
This huge amount of structural data needs very fast and accurate computer programs to deal
with extracting structural information, and comparing new proteins with previously 
annotated ones.
Those programs should enable not only
structure-to-structure search, but also alignment over all proteins from the
database in a real time.
So far typical computations done has been performed by several state-of-the-art methods,
including \prog{} code.
Because of the overwhelming and constantly growing 
amount of processing to be performed, scientists requested
the support of the \emph{Joint Cell Competence Centre} \cite{jccc}.
Due to an ongoing collaboration between \icm{} and \ibm{}, it was decided to
port the \prog{} code, inter alia, to the PowerXCell8i Architecture and use Cell
based machines as a computational facility.\\
The purpose of this article is to present how the \prog{} program has been
ported and tuned on novel architectures and how processing performance of
accelerated versions compares to the x86 implementation.

%%%%%%%%%%%%%%%%%%%%%%%%%%%%%%%%%%%%%%%%%%%%%%%%%%%%%%%%%%%%%%%%%%%%%%%%%%%%%%%%%%%%%%%%%%%%%%%%%%%%%%%%%%%%
\section{Overview of the algorithm}
%%%%%%%%%%%%%%%%%%%%%%%%%%%%%%%%%%%%%%%%%%%%%%%%%%%%%%%%%%%%%%%%%%%%%%%%%%%%%%%%%%%%%%%%%%%%%%%%%%%%%%%%%%%%
\prog{} program provides the structural alignment of two proteins.
It uses in-house customised version of Smith-Waterman dynamic programing
algorithm combined with intensive three-dimensional rotation and translation
routines that align two geometrical objects in order to minimize the root mean square
distance computed for all heavy atoms from both proteins.
The pseudo-code of entire program is listed in the Algorithm 1 section.
Virtually, the most time consuming part of the code is the preparation of structural
alignment matrix, which is input data for the Smith-Waterman algorithm (step 2 of Algorithm 1).
To identify similarity substructures of two proteins the program compares
parts of their chains with a fixed length of 256 amino acids.
We call them \emph{segments}.
Program analyses structural similarity between each pair of segments as follows.\\
First of all it decides whether central parts of segments, which we call
\emph{seeds}, are similar enough to proceed with further computations.
Seeds are very short subsequences of chains with the length of 13 amino acids.
Algorithm makes rotations and translations of the Calpha atoms of both seeds in
order to minimize the root mean square deviation (RMSD) between them.
Secondly, if the structural similarity of the two seeds is high enough,
the algorithm starts to analyze two longer continuous parts of the main chains
centered on seeds.
It uses a rotation matrix and a translation vector for a Cartesian-space
superimposition of the two seeds to rotate and translate these large segments.
Next, it defines the similarity matrix for the dynamic programming in the
following way.
If two Calpha atoms taken from the superimposed large segments are close enough
in space (below 3A), it assigns 1 as their similarity score, 0 is taken
otherwise.
Then alignment score based on such similarity score matrix is computed.
If it is high enough, algorithm passes this pair of segments
on to the next filter.
The whole procedure is repeated for the subsequences of length 100, 200 and 256
amino acids centered on seeds.
The resulting score (number of superimposed Calpha atoms) is recorded in an
additional final alignment matrix, containing segment superposition scores.
This matrix is then used to find the best alignment of whole proteins.
If no pair of segments has passed all filters, the overall score is 0.

\begin{algorithm}
\caption{A~sketch of the structural alignment algorithm}

\begin{tabular}{lp{13.39cm}}
{\em Input:} &
Two protein sequences: \texttt{protein1} and \texttt{protein2}.\\
{\em Output:} &
Structural alignment of the given proteins.\\
\end{tabular}

\begin{enumerate}
\item
Divide each protein into segments of fixed length (256 amino acids).

\item 
\begin{verbatim}
// Create a structural alignment matrix containing similarity
// scores between each pair of segments.
\end{verbatim}
For each \texttt{segment1} in \texttt{protein1} and \texttt{segment2} in \texttt{protein2},
compute the number of superimposed Calpha atoms in the aligned pair of segments:
\begin{verbatim}
    similarityMatrix[segment1, segment2] =
        ComputeAlignmentOfSegments(segment1, segment2)
\end{verbatim}

\item
Find alignment of the whole proteins by running Smith-Waterman algorithm on them with the \texttt{similarityMatrix} as an input.

\end{enumerate}
\end{algorithm}

\begin{algorithm}
\caption{\texttt{ComputeAlignmentOfSegments(segment1, segment2)}}

\begin{tabular}{lp{13.39cm}}
{\em Input:} &
Two segments of potein sequences.\\
{\em Output:} &
The number of superimposed Calpha atoms in the aligned segments.\\
\end{tabular}

\begin{enumerate}
\item
Extract the seed of each segment (seed is the central part of segment consisting of 13 amino acids).

\item
Find the rotation matrix and the translation vector minimizing the RMSD between the two seeds.

\item
If the optimal RMSD is too low, exit and return 0.

\item
For \texttt{len} in \{100, 200, 256\}:
\begin{itemize}
\item[(4a)]
Define the subsequences to consider:\\
\begin{tabular}{lp{8cm}}
\texttt{subsequence1 =} &
the subsequence of length \texttt{len} surrounding the seed of \texttt{segment1}. \\
\texttt{subsequence2 =} &
the subsequence of length \texttt{len} surrounding the seed of \texttt{segment2}. \\
\end{tabular}

\item[(4b)]
Align the two subsequences using the rotation matrix and the translation vector computed in the previous step.

\item[(4c)]
\begin{verbatim}
// Define the alignment matrix for the two subsequences:
\end{verbatim}
For each \texttt{atom1} in \texttt{subsequence1} and \texttt{atom2} in \texttt{subsequence2}:
\begin{verbatim}
    alignmentMatrix[atom1, atom2] = 
        distance(atom1, atom2) < 3A ? 1 : 0
\end{verbatim}

\item[(4d)]
Find the alignment of subsequences by running Smith-Waterman algorithm on them with the \texttt{alignmentMatrix} as an input.

\item[(4e)]
If computed alignment is better than the one found previously, continue the loop. Otherwise, return the best alignment found previously.

\end{itemize} 

\end{enumerate}
\end{algorithm}



%%%%%%%%%%%%%%%%%%%%%%%%%%%%%%%%%%%%%%%%%%%%%%%%%%%%%%%%%%%%%%%%%%%%%%%%%%%%%%%%%%%%%%%%%%%%%%%%%%%%%%%%%%%%
\section{Porting process}
%%%%%%%%%%%%%%%%%%%%%%%%%%%%%%%%%%%%%%%%%%%%%%%%%%%%%%%%%%%%%%%%%%%%%%%%%%%%%%%%%%%%%%%%%%%%%%%%%%%%%%%%%%%%
After profiling the original version of \prog{} program we found out that
the most time consuming part of the code was a function implementing algorithm
which compares two given segments.
Intrinsically, it was executed in sequence with each pair of segments as a
parameter.
As a result, program spent more than $90\%$ of its time in that function.
We decided to accelerate that part of program by parallelization based on two
programming models: libspe2 library model for Cell and OpenMP model for x86
architecture.

\subsection{Parallel scheme}
Computations for each pair of segments can be performed independently with
no communication occurring between working threads.
Nevertheless, at the end the main thread must collect all partial results for
each pair of segments and compute final result based on them.
We wondered whether those extra computations performed on the partial results
would cause a new bottleneck.
In the case of Cell implementation computing of the final result
is handled by PPU.
In the case of OpenMP version this is done by the master thread.
Fortunately, it turned out in the end that accelerated versions of \prog{}
program spends only a fraction of percent of their execution times computing
final result.

\subsection{PowerXCell8i Implementation}
We used the Cell SDK and its SPE library to implement this simple parallel
scheme.
The PPE processor executes the main thread of application.
Its role is to read and preprocess description of two protein sequences, create
SPE contexts, run appropriate working threads, collect partial results and
finally compute and return the final score and alignment.
Each SPE processor gets its set of pairs of segments.
Than, consecutively for each pair of segments from the set, it loads the
descriptions of two segments into local memory space (Local Store),
executes comparing function and stores the result back into the main memory.

\subsubsection{Memory issue}
Porting most of the applications to SPE processor is rather challenging task
because of the limitation of size of Local Store, which is only 256 KB.
This small space should be wisely administrated, because it must accommodate
both program instructions and operating data.
Moreover, the only way to load and store data to and from Local Store is DMA
mechanism.
It gives programmer a great deal of control, but on the other hand it is not
simple to use.\\
The \prog{} code is unfortunately memory consuming.
In the original version each execution of comparing algorithm requires memory
for at least $256^2$ floating point numbers and $256^2$ short integer values.
It is of course too much for the Local Store.
That is why we had to perform some memory optimizations including compression
and introduction of bit operations.
The resulting code was slightly slower and less accurate than the original one,
but it is an even trade.
Thanks to those sacrifices we were able to fit our program to Local Store.

\subsubsection{SIMD optimizations}
According to the profile, the original version of \prog{} program spends nearly
$30\%$ of its execution time calculating rotation matrices and translation
vectors for Cartesian-space superimposition of pairs of segments.
The part of code responsible for that task turned out to be a good candidate for
a vectorization process due to the quantity of simple algebraic operations.\\
First of all, we decided to take advantage of \texttt{GCC} compiler auto
SIMDizing abilities by switching on option \texttt{-ftree-vectorize}.
At the beginning it did not help much since the code occurred to be to complex
to automatic analysis for the compiler.
Therefore we had to make some simplifications.
We followed the guidelines proposed in \cite{rb}. \\
Even after using specific compiler directives compiler was unfortunately unable
to automatically vectorize some of the loops.
We decided to tune the remaining parts of code manually by instruction
substitution.
The main vector operations used in the SPE computational kernel were
\texttt{spu\_add}, \texttt{spu\_mul}, \texttt{spu\_madd} and
\texttt{spu\_splats}.
All of these instructions were operating on float 4 entry vectors, so we could
speed up the vectorized loops of the appropriate steps of the algorithm about
4 times.
The result of this effort is presented in Tab.~\ref{tab:t1}.

\begin{table}[htb]
\begin{footnotesize}
\caption{Performance results of SIMD and noSIMD versions.}
\label{tab:t1}
\newcommand{\m}{\hphantom{$-$}}
\newcommand{\cc}[1]{\multicolumn{1}{c}{#1}}
\renewcommand{\tabcolsep}{0.5pc} % enlarge column spacing
\renewcommand{\arraystretch}{1.2} % enlarge line spacing
\begin{tabular}{@{}llll}
\hline
\textbf{Version} & \textbf{Compiler} & \textbf{Average time} & \textbf{Speedup} \\
\hline
% AMD Opteron, noSIMD, 1 core & \texttt{GCC} (no optimizations) & 1.435 s & X \\
PowerXCell8i, noSIMD, 1 SPE & \texttt{GCC} (\texttt{-O3}) & 1.546 & 1.0 \\
PowerXCell8i, noSIMD, 16 SPEs & \texttt{GCC} (\texttt{-O3}) & 0.181 s &  8.541 \\
PowerXCell8i, SIMD, 16 SPEs & \texttt{GCC} (\texttt{-O3 -ftree-vectorize}) & 0.164 s & 9.427 \\
\hline
\end{tabular}\\[2pt]
\end{footnotesize}
\end{table}

\subsubsection{Efficient implementation of the parallel scheme}
In our first approach to implement chosen parallel scheme we met very
interesting problem.
At the beginning we were assigning jobs for the computational kernels
arbitrarily.
Each SPE program was given a consecutive sequence of pairs of segments to
operate on.
Nevertheless, it was wrong decision because of the inefficient load balancing.
As described before, comparing algorithm is sophisticated and it does not always
behave in the same manner.
For example, if it discovers that seeds are not similar enough, it finishes
computations very quickly.
Moreover, cutoffs may occur at the every stage, including analysis of longer
segments.
That is why it is important to assign each SPE with more or less the same
amount of real work, which may not necessarily mean the same amount of pairs of
segments.\\
We tried many different ways to fulfil this requirement.
First of all, we decided to divide the whole set of tasks into 16 random subsets
on the PPE side of application.
Each SPE program operated on one of those randomly chosen kits.
This solution on the one hand allowed us to achieve very good load balancing,
but on the other hand drastically slowed down PPE thread.
It took about $15\%$ of its execution time to randomly permute set of tasks.
We could not afford such a waste.\\
The next idea, which came to our minds, was to use PPE thread as a management
resource distributing workload coherently with computations.
In this approach each idle SPE asks PPE for a new commission.
We implemented and tested a few versions taking advantage of SPE mailboxes
as well as advanced DMA transfers with double buffering.
Unfortunately non of them met our expectations, because of the slowdown caused
by the communication process.\\
Finally, it turned out that the simplest solution is the best.
Instead of randomly distributing the set of commissions on the PPE side, we have
chosen an arbitrary random permutation and embed it into the SPE program as a
constant.
In this approach each SPE kernel is statically assigned a random set of tasks.
It allowed us to completely eradicate communication between PPE and SPE and
achieve reasonable load balancing.
The performance comparison of all of these methods is presented in
Tab.~\ref{tab:t2}.

\begin{table}[htb]
\begin{footnotesize}
\caption{Performance results of various versions.}
\label{tab:t2}
\newcommand{\m}{\hphantom{$-$}}
\newcommand{\cc}[1]{\multicolumn{1}{c}{#1}}
\renewcommand{\tabcolsep}{0.5pc} % enlarge column spacing
\renewcommand{\arraystretch}{1.2} % enlarge line spacing
\begin{tabular}{@{}lll}
\hline
\textbf{Version} & \textbf{Compiler} & \textbf{Average time} \\
\hline
% Original, AMD Opteron, 1 core & \texttt{GCC} (no optimizations) & 1.435 s \\
Randomization on PPE, 16 SPEs & \texttt{GCC} (\texttt{-O3 -ftree-vectorize}) & 0.247 s \\
Dynamic distribution, 16 SPEs & \texttt{GCC} (\texttt{-O3 -ftree-vectorize}) &  0.370 s \\
Static permutation, 16 SPEs & \texttt{GCC} (\texttt{-O3 -ftree-vectorize}) & 0.164 s \\
\hline
\end{tabular}\\[2pt]
\end{footnotesize}
\end{table}

\subsubsection{Other optimizations}
Each SPE program is executed once at the beginning and serves as a computational
facility for many tasks.
Thanks to that we are able to eliminate time needed for SPE context creation.
In addition we decided to make use of DMA double buffering
mechanism.
While SPE is carrying on computations, its Memory Controller can coherently load
the next portions of data from the main memory.
This simple idea let us hide input-output operations behind real computations.\\
In comparison to the single SPE, we achieved an overall speedup of 6.14 while
executing on 8 SPEs and 9.39 while executing on 16 SPEs.
We decided to put into use one more improvement.
We run two parallel instances of \prog{} program, each using 8 SPEs, on the
QS22 server equipped with two Cell chips.
That gives as an average speedup of 12.

\subsection{OpenMP Implementation}
Implementing chosen programming model using OpenMP occurred to be very
simple.
The whole set of pairs of segments is dynamically (and automatically)
distributed among number of OpenMP threads.
Each such a thread executes an algorithm similar to the one described in the 
PowerXCell8i's section.
It gets a pair of segments, computes an answer and stores the result in the
specified place in the main memory.\\
In spite of its simplicity, the OpenMP implementation of our application
occurred to be very efficient and accurate.
In comparison to the single-threaded version, we achieved an overall speedup of
4.37 while executing on 8 cores.
We take advantage of the same schema as the one used with PowerXCell8i
processors. We run two parallel instances of \prog{} program, each using 8
threads, on blade equipped with 16 processor units.
That gives as an average speedup of 8.5 over the initial x86 version.

%%%%%%%%%%%%%%%%%%%%%%%%%%%%%%%%%%%%%%%%%%%%%%%%%%%%%%%%%%%%%%%%%%%%%%%%%%%%%%%%%%%%%%%%%%%%%%%%%%%%%%%%%%%%
\section{Performance results}
%%%%%%%%%%%%%%%%%%%%%%%%%%%%%%%%%%%%%%%%%%%%%%%%%%%%%%%%%%%%%%%%%%%%%%%%%%%%%%%%%%%%%%%%%%%%%%%%%%%%%%%%%%%%
\subsection{PowerXCell8i code analysis}
We have used a \texttt{spu\_timing} facility to analyze and tune computational
kernels of \prog{} Cell implementation.
Results presented in Tab.~\ref{tab:t3} show that we achieved
very good scaling over increasing number of working SPE threads.
The part of code, which can be parallelized, takes about $97.994\%$ of execution
time of single-threaded version of application.
Therefore, according to the Amdahl's Law, the best theoretical speedup with 16
working threads, is approximately
$\frac{1}{(1-0.97994) + \frac{0.97994}{16}}\approx12$.
Our result is slightly worse.
We achieve an overall speedup of almost 9.
The most probable cause is using built-in permutation of commissions instead of
actually randomly permutated set of segments.
This approach increases probability of occurrence of inefficient load balancing.
\begin{table}[htb]
\begin{footnotesize}
\caption{Profile results.}
\label{tab:t3}
\newcommand{\m}{\hphantom{$-$}}
\newcommand{\cc}[1]{\multicolumn{1}{c}{#1}}
\renewcommand{\tabcolsep}{0.5pc} % enlarge column spacing
\renewcommand{\arraystretch}{1.2} % enlarge line spacing
\begin{tabular}{@{}llll}
\hline
\textbf{Part of code} & \textbf{1 SPE} & \textbf{8 SPEs} & \textbf{16 SPEs} \\
\hline
Preparing data & 0.004782 \% & 0.142613 \% & 0.359532 \% \\
Creating  SPE threads & 0.121318 \% & 7.627334 \% & 21.029137 \% \\
Waiting for threads & 97.844704 \% & 80.333420 \% & 62.397312 \% \\
Computing global result & 0.019680 \% & 0.139842 \% & 0.219315 \% \\
\hline
\end{tabular}\\[2pt]

\bigskip

\begin{tabular}{@{}llll}
\hline
\textbf{Part of code} & \textbf{1 SPE} & \textbf{8 SPEs} & \textbf{16 SPEs} \\
\hline
Preparing data & 0.000051 s & 0.000247 s & 0.000466 s \\
Creating  SPE threads & 0.001170 s & 0.012919 s & 0.027042 s \\
Waiting for threads & 1.492583 s & 0.207126 s & 0.114413 s \\
Computing global result & 0.000527 s & 0.000571 s & 0.000619 s \\
\hline
\end{tabular}\\[2pt]
\end{footnotesize}
\end{table}

\subsection{Performance Comparison}

We have designed test to examine operational performance of \prog{} code
executing on various architectures.
We have chosen a set of 18 proteins from a database and compared execution times
on an Quad-Core AMD Opteron Processor 8354 based nodes and QS22 server.
Each pair of proteins from the test set was compared during a test run, which
gave us 324 single test cases.
The average results are presented in Tab.~\ref{tab:t4}.
\begin{table}[htb]
\begin{footnotesize}
\caption{Performance results on two systems: AMD Opteron 8354 and PowerXCell8i QS22.}
\label{tab:t4}
\newcommand{\m}{\hphantom{$-$}}
\newcommand{\cc}[1]{\multicolumn{1}{c}{#1}}
\renewcommand{\tabcolsep}{0.5pc} % enlarge column spacing
\renewcommand{\arraystretch}{1.2} % enlarge line spacing
\begin{tabular}{@{}llll}
\hline
\textbf{Architecture} & \textbf{Compiler} & \textbf{Time} & \textbf{Speedup} \\
\hline
AMD, 1 OpenMP thread & \texttt{GCC} (\texttt{-O3}) & 116.85 s & 1.00 \\
AMD, 4 OpenMP threads & \texttt{GCC} (\texttt{-O3}) & 38.22 s & 3.05 \\
AMD, 8 OpenMP threads & \texttt{GCC} (\texttt{-O3}) & 26.72 s & 4.37 \\
AMD, 16 OpenMP threads & \texttt{GCC} (\texttt{-O3}) & 26.09 s & 4.47 \\
\hline
PowerXCell8i, 1 SPE & \texttt{GCC} (\texttt{-O3 -ftree-vectorize}) & 499.07 s & 1.00 \\
PowerXCell8i, 8 SPEs & \texttt{GCC} (\texttt{-O3 -ftree-vectorize}) &  81.20 s & 6.14 \\
PowerXCell8i, 16 SPEs & \texttt{GCC} (\texttt{-O3 -ftree-vectorize}) & 53.14 s & 9.39 \\
\hline
\end{tabular}\\[2pt]
\end{footnotesize}
\end{table}
As we can see, Cell accelerated version of \prog{} occurred to be approximately
two times slower than the OpenMP code executed on blades equipped with AMD
processors.
Those performance differences could be caused by disparities in technical
parameters of chosen machines.
According to our experiments, multiplication of two floating point scalars is
almost two times slower on PowerXCell8i SPE processors than on AMD Opterons.
Unfortunately a great deal of \prog{} code could not be vectorized and, as a
result, arithmetic operations on scalars are intense in our code.
Moreover, the PowerXCell8i version is slightly more complex.
For example, necessity of performing memory optimisations, described in one of
previous sections, has its own overhead.
Another very important feature that kills the performance of the implemented SPU
kernels is a big number of branches introduced by the algorithm and a lack of
branch prediction mechanism within the SPU architecture.
As a result one of the most important computational parts of the code,
Smith-Waterman algorithm, is significantly slowed down on Cell architecture.
Taken together, all those factors cause the drop-down of performance.\\
On the other hand, PowerXCell8i implementation has one desired property, which
is unfortunately not a feature of OpenMP-based program, namely --- very good
scalability.
It is probably due to the limited memory bandwidth.
When 16 working threads try to simultaneously read data from shared memory
located on their blade, they experience significant slowdown of data
transfer.
It should be stated here that the comparison of two longest sequences in the
benchmark test takes approximately only 0.22 seconds.
The memory bandwidth  is not a problem in PowerXCell8i implementation, due to
the presence of highly efficient Element Interconnect Bus (EIB), which provides
each SPE and its memory controller with private and very fast connection to the
main memory.

%%%%%%%%%%%%%%%%%%%%%%%%%%%%%%%%%%%%%%%%%%%%%%%%%%%%%%%%%%%%%%%%%%%%%%%%%%%%%%%%%%%%%%%%%%%%%%%%%%%%%%%%%%%%
\section{Summary}
%%%%%%%%%%%%%%%%%%%%%%%%%%%%%%%%%%%%%%%%%%%%%%%%%%%%%%%%%%%%%%%%%%%%%%%%%%%%%%%%%%%%%%%%%%%%%%%%%%%%%%%%%%%%
The current evaluation of \prog{} performed on dataset of circa 300 query proteins
reveals the quality of the tool, as compared with other programs. When compared
with DALI server, our tool is able to generate similar number of correct models, however
the final alignment quality is better in the case of the second service. In the 
case of distant structural comparisons, our method gets better ratings in all categories
(such as specificity analysis) when MaxSub evaluation method is used.
Concluding, the \prog{} software is on average less sensitive than the DALI
server, yet it is better than CE or VAST tools.\\
On the side of optimizalization of the core, we have ported the \prog{} 
program to the novel high-performance
architectures and achieved very good speedup.
Our accelerated programs are planed to be embedded into the web application,
which will allow very fast and accurate mechanism for structural alignment of
proteins.
By taking advantage of PowerXCell8i and Quad-Core x86 novel architectures,
we were able to significantly reduce time of single computation and as a result
provide scientists all over the world with always up-to-date and fast tool for
structural alignment experiments.
The present version of our algorithm is very fast. It takes circa 2 seconds 
for a query protein of 500 amino acids to scan a database of 1000 templates.
The single comparison of two proteins with size circa 500 amino acids 
is performed within a runtime of 0.002 seconds for \prog{}, where 
older version of the software took around 0.017 seconds. Other structural
alignment programs are significantly slower, CE takes 3 seconds to compare two
typical proteins, LGScore2 is around 6 seconds. \\
Because of its speed and portability (the source code is available from authors 
upon request) we believe that \prog{} program will continue to be widely used
in structural genomics projects, improving the structural comparison of  newly crystallized
proteins with large structural databases. We are planning to provide the internet web
server interface to the PDB database, in order to give user access to rapid structural 
alignment of its protein of interest with three-dimensional crystals or 3D models
of proteins. 
%%%%%%%%%%%%%%%%%%%%%%%%%%%%%%%%%%%%%%%%%%%%%%%%%%%%%%%%%%%%%%%%%%%%%%%%%%%%%%%%%%%%%%%%%%%%%%%%%%%%%%%%%%%%

\section{Acknowledgments}
%%%%%%%%%%%%%%%%%%%%%%%%%%%%%%%%%%%%%%%%%%%%%%%%%%%%%%%%%%%%%%%%%%%%%%%%%%%%%%%%%%%%%%%%%%%%%%%%%%%%%%%%%%%%
This work was supported by EC OxyGreen (KBBE-2007-212281) 6FP project as
well as the Polish Ministry of Education and Science (N301 159735,
and others).

\begin{thebibliography}{99}

\bibitem[1]{nau} Nautilus system,
http://cell.icm.edu.pl/index.php/Nautilus

\bibitem[2]{g500} The Green500 List,
http://www.green500.org/lists/2009/06/list.php

\bibitem[3]{3dhit1} Plewczynski D, Pas J, von Grotthuss M, Rychlewski L.
\textit{3D-Hit: fast structural comparison of proteins.}
Appl Bioinformatics. 2002;1(4):223-5.

\bibitem[4]{3dhit2}  Plewczynski D, Pas J, Von Grotthuss M, Rychlewski L.
\textit{Comparison of proteins based on segments structural similarity.}
Acta Biochim Pol. 2004;51(1):161-72.

\bibitem[5]{chle} Chothia C, Lesk AM. \textit{The relation between the
divergence of sequence and structure in proteins.} EMBO J.; 5: 823-6 (1986).

\bibitem[6]{buj2000} Bujnicki JM. \textit{Phylogeny of the restriction
endonuclease-like superfamily inferred from comparison of protein structures.}
J Mol Evol.; 50: 39-44 (2000).

\bibitem[7]{jms} Johnson MS, Sutcliffe MJ, Blundell TL. \textit{Molecular
anatomy: Phyletic relationships derived from three-dimensional structures of
proteins.} J Mol Evol.; 30: 43-59 (1990).

\bibitem[8]{pdb} Berman HM, Westbrook J, Feng Z, Gilliland G, Bhat TN,
Weissig H, Shindyalov IN, Bourne PE. \textit{The Protein Data Bank.}
Nucleic Acids Res.; 28: 235-42 (2000).

\bibitem[9]{rb} Abraham Arevalo, Ricardo M. Matinata, Maharaja Pandian,
Eitan Peri, Kurtis Ruby, Francois Thomas, Chris Almond,
\textit{Programming the PowerXCell8i Architecture, Examples and Best Practices},
RedBooks, IBM (2008)

\bibitem[10]{jccc} IBM, ICM (University of Warsaw):
Joint Cell Competence Center, http://cell.icm.edu.pl




\end{thebibliography}

\end{document}
